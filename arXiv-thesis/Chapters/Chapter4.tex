
\chapter{Conclusion} 
\label{Chapter4} 
\lhead{Chapter 4. \emph{Conclusion}} 

After studying the current research around WebSocket in a distributed
environment, this thesis focused on benchmarking node.js's real time
engine SocketCluster.

SocketCluster is a promising library still actively under development.
It efficiently provides a highly scalable WebSocket server that makes
use of all available cpu cores on an instance. It removes the limitation
of having to run node.js code on single cores.

\textbf{Experiment conclusion}

Experiments carried out on SocketCluster revealed two main limitations.  If
running on comparable hardware, a SocketCluster worker will be less efficient
then a basic engine.io implementation. Also SocketCluster efficiency
dramatically drops if it is run with more process then available cores because
of context switching.

SocketCluster should be used in highly parallel environment and
therefore these limitations rarely apply. SocketCluster theoretically  enables
user to scale an application vertically  without limits. N being the number of
cores the server has, SocketCluster has been proved to be at least
$\frac{N}{2}$ more efficient then a basic node.js implementation.  As the
number of cores rises, it looks like the performance could be slightly better
then $\frac{N}{2}$. The load balancers begins to misbehave and performance
is limited by a few overloaded load balancers. However, it is probably only a
question of time until a patch fixes this issue.

\newpage
\textbf{Future work}

While benchmarking SocketCluster, useful SocketCluster features were
considered.  System administrators could benefit from a real-time monitoring tool
to check the state of each threads and thus help them manage the size of the
cluster. The monitoring tool could even be linked with an algorithm to
automatically append or delete threads.  SocketCluster would then be an autonom
entity. Scaling on its own without any human interaction.

Also further studies could be carried on SocketCluster on more then on server.
Since each cores already operates as a separate thread, the perforance
shouldn't decrease if spread on many servers. But it might be worth checking.
















