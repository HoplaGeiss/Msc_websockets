% \label{Chapter1}
\chapter*{Introduction}
\lhead{\emph{Introduction}}

\textbf{Problem statement}

WebSockets are implemented in a wide range of applications. As a result, a lot
of different languages and specific libraries have been specifically developed
for WebSockets.

The first part of this paper is an introduction to the WebSocket protocol and
on the different implementation options when building a WebSocket cluster. In a
second part, it focuses on the new real-time engine: SocketCluster and makes a
benchmark of this library.

\textbf{Thesis structure}

The first chapter is a literature review. The goal is to inform the reader
about the WebSocket protocol and to go through the different WebSocket
implementations. Therefore studying scalability and heterogeneous
implementations.

The second chapter is an introduction to the experiments. It is dedicated to the
design and the implementation of the infrastructure used later on. It mostly 
introduces the benchmark library used.

The experiment chapter is a comprehensive benchmark of SocketCluster. It 
compares the performances to a classic engine.io implementation and also
studies the limitations of the library.

To finish the last part concludes this thesis and suggest future work on 
SocketCluster.


