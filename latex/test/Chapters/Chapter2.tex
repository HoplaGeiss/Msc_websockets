
\chapter{Design and Implementation} 
\label{Chapter2} 
\lhead{Chapter 2. \emph{Design and implementation}} 

Current research around WebSocket is centered around distributed computing. Either
on CPUs architecture, GPUs architecture or heterogeneous architecture. For the
time being, clustering WebSocket servers is rather difficult and reserved to
researcher or specialized companies. Actually in Node.js, there is hardly any
library to simply and efficiently implement a multi-core server.  Node.js
single thread nature is a double edge sword. On one side it allows more
concurrent connections to be established but it also means special attention
needs to be given to run the code on all the servers cores. SocketCluster is a
brand new real-time engine aiming exactly at that.  At this point of my thesis
I had to make a choice between either the theoretical study of WebSocket
clusters or the benchmarking of SocketCluster. After contacting Jonathan
Gros-Dubois, the creator of SocketCluster, I made up my mind for the latter.
Indeed, SocketCluster being under development the tests carried out so far are
rather sparse. 

\section{SocketCluster library}

As described on the github project  \citep{Reference38}, SocketCluster is a
fast, highly scalable HTTP and WebSocket server. It facilitates the creation of
multi-process real-time application that make use of all CPU cores on a
machine/instance. Therefore removing the limitations of having to run a Node.js
server as a single thread.  SocketCluster's focus is on vertical scaling. If N
is the number of cores available on the server, then SocketCluster is N time
than any comparable single-threaded WebSocket server. Under the hood, the
application deploys itself on all available cores as a cluster of process. The
process can be grouped in three categories: stores, workers and load balancers.

\section{Challenges encountered using SocketCluster}

At first my study of SocketCluster was far from satisfactory. Past a
total of 512 communications channel, new sockets were inexplicably crashing.

\textbf{U-limit}

This comes from a system limit set up on linux operating systems. By default
the maximum number of file that can be sent over tcp is 1024. 

Fortunately, this limit can be increased by appending this line: \texttt{ubuntu
soft nofile "number of file"} in \texttt{/etc/security/limits.conf}

Once this problem was fixed I looked into a benchmark to carry out, Jonathan Gros-Dubois
advised me to focus on concurrent connections tests.

\textbf{C 10K Problem}

The C 10K is an historic challenge issued in 1999 by Dan Kegel. It consist of
reaching 10 000 concurrent client connections. Engineers solved this problem by
fixing operating systems kernel and creating new single threaded programming
languages like Node.js. 

Therefore one of my objectives while testing SocketCluster was to see how many
concurrent connections it can handle.

However this should not be a problem for this library, contemporary objective is
rather to achieve 10 000 000 concurrent connections like mentioned in the
excellent article in highscalability.com \citep{Reference39}. Such amount of
connections is beyond the scope of this thesis, but apparently the solution to
improve the number of connections is to move heavy lifting from the kernel to
the application itself.

Another topic to consider before begin testing was how to monitor the
application.

\section{Monitoring tool}

Monitoring tools can be divided in two categories. Basic Real time monitoring
and more convenient tool saving statics in spreadsheets and eventually
even directly plotting graphs. Most of them can be configured to record
processor load on each cores. But ideally SocketCluster tests would
require to record each threads load's. This way, if run less process than
available cores are being run the exact usage of each thread can still be 
found.

For this reason and also to have more freedom on how data is being
processed, out of the box tool have been cast aside for more basic tools like top 
and htop. htop has been used to visualize data in real time and check if the 
test was running flawlessly. top has been used in batched mode to output
the data in files.

In a latter phase, bash operation is used to format the raw data extracted from 
top's file. And finally, graphs are plotted with gnuplot.









