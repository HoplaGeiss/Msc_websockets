\chapter{Conclusion} 
\label{Chapter5} 
\lhead{Chapter 5. \emph{Conclusion}} 

After studying the current landscape of research around WebSocket 
in a ditributed environment, this thesis focuses on the benchmarking 
of the node.js's real time engine SocketCluster.\\

SocketCluster is a promising library still actively under development.
It efficiently provides a highly scalable WebSocket server that make
use of all available CPU cores on an instance. It removes the limitation
of having to run Node.js code on single cores.\\

Experiments carried out on SocketCluster revealed two main limitations.  If
running on comparable hardware, a SocketCluster worker will be less efficient
then a basic engine.io implementation. Also SocketCluster efficiency
dramatically drops if run with more process then available cores because of
context switching.\\

Anyway, SocketCluster should be used in highly parallel environment and
therefore these limitations rarely appply. SocketCluster theoriticaly  enables
user to scale an application vertically  without limits. N being the number of
cores the server has, SocketCluster has been prooved to be at least
$\frac{N}{2}$ more efficient then a basic node.js implementation.  As the
number of cores rises, it looks like the performance could be slightly better
then $\frac{N}{2}$. The load balancers begins to missbehave and performances
are limitted by a few overloaded load balancers. However, it is probably only a
question of time until a patch fixes this issue.\\

While benchmarking SocketCluster, useful SocketCluster features were
considered.  System administrator could benefit from a realtime monitoring tool
to check the state of each threads and thus help them manage the size of the
cluster. The monitoring tool could even be linked with an algorithm to
automatically append or delete threads.  SocketCluster would then be an autonom
entity. Scaling on its own without any human interaction.\\



